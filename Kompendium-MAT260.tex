\documentclass[a4paper, 12pt]{article}

\usepackage[utf8]{inputenc}
\usepackage[norsk]{babel}
\usepackage{graphicx}
\usepackage{amsfonts}
\usepackage{amsmath}
\usepackage[unicode]{hyperref}

% Enumeration
\usepackage{enumerate}
\usepackage{enumitem} 
\usepackage[sharp]{easylist}
\usepackage{multicol}

%For equation reference
\usepackage{amsmath}
\usepackage{amsthm}
\usepackage{amssymb}
\usepackage{commath}

%More maths stuff
\usepackage{amsfonts}

% Change the margins
\usepackage{geometry}
\geometry{a4paper, top=2cm, left=3cm, right=3cm, bottom=2cm, includehead, includefoot}
%opening
\title{Compendium MAT260}
\author{Anita Gjesteland}

\begin{document}

\maketitle




\section{Preliminaries}
\subsection{Norms}

\newtheorem{mydef}{Definition}
\newtheorem{myex}{Example}
\newtheorem{mythm}{Theorem}
\begin{mydef}
	Let V be a linear spave over $\mathbb{R}$. A function $\norm{\cdot}: V \rightarrow$ $\mathbb{R}$ is a norm on V if it satisfies the following:\\
	\begin{enumerate}[label=(\roman*)]
		\item $\norm{\lambda x} = \abs{\lambda}\norm{x}$ \: $ \forall x \in \mathbb{R} $
		\item $\normalcolor{x + y} \leq \norm{x} + \norm{y}$ \:$ \forall x,y \in V$
		\item $\norm{x} = 0 \Leftrightarrow x = 0$
	\end{enumerate}
\end{mydef}

\begin{myex} 
	p - norm: $\norm{x}_p = (\sum_{i = 1}^{n}\abs{x_i}^p)^{1/p}$
	\begin{enumerate}
		\item[p = 2:] $\norm{x}_2 = (\sum_{i = 1}^{n}\abs{x_i}^2)^{1/2} $
		\item [p = $\infty$:] $\norm{x}_\infty = \underset{i \in \{1,...,n\}}{\max \abs{x_i}}$
	\end{enumerate}
\end{myex}

\subsubsection{Matrix norms}

A matrix norm has the following properties:
\begin{enumerate}[label=(\roman*)]
	\item $\norm{\lambda A} = \abs{\lambda}\norm{A} $\: $\forall \lambda \in \mathbb{R},$ \:$ \forall A \in \mathbb{R^{n,n}}$
	\item $\norm{A + B} \leq \norm{A} + \norm{B}$ \: $\forall A,B \in \mathbb{R^{n,n}}$
	\item $\norm{A} = 0 \rightarrow A = 0_n$
	\item $\norm{AB} \leq \norm{A}\norm{B}$ \: $\forall A,B \in \mathbb{R^{n,n}}$
\end{enumerate}

\begin{myex}
	\begin{enumerate}
		\item $\norm{A}_1 = \underset{x \neq 0}{\sup} \frac{\norm{Ax}_1}{\norm{x}_1} = \underset{j=1,...,n}{\max} \sum_{i=1}^{n}\abs{a_{ij}}$. Max column sum
		\item $\norm{A}_\infty = \underset{x \neq 0}{\sup} \frac{\norm{Ax}_\infty}{\norm{x}_\infty} = \underset{i = 1,...,n}{\max} \sum_{j=1}^{n} \abs{a_{ij}}$. Max row sum
		\item $\norm{A}_2 = \underset{x \neq 0}{\sup} \frac{\norm{Ax}_2}{\norm{x}_2}$. Also: $\sqrt{\rho(A^TA)}$, $\rho$ is the spectrum of A. Euclidean norm
		\item $\norm{A}_F = (\sum_{i}^{n}\sum_{j}^{n}\abs{a_{ij}}^2)^{1/2}$. Frobenius norm
	\end{enumerate}
\end{myex}

\subsubsection{Function norm}

Function norms we have used in this course are among others: \\
\begin{enumerate}
	\item $\norm{f}_p = (\int_{a}^{b} \abs{f(x)}^2\omega (x) dx)^{1/p}$ where $\omega(x)$ is some weight function.
	\item $\norm{f}_\infty = \underset{a \leq x \leq b}{\max} \abs{f(x)}$
\end{enumerate}

\subsection{Banach stuff}
\begin{mydef}[Banach space]
	A Banach space is a normed space that is complete. I.e every Cauchy sequence is convergent (to an element of the space).
\end{mydef}

\begin{mythm}
	Let $(X,\norm{\cdot})$ be a Banach space and let $U \subseteq X $ be a subset of X and $f: U \rightarrow X$ be a function. If 
	\begin{enumerate}[label = (\roman*)]
		\item U is closed
		\item f is a contraction
		\item $f(U) \subseteq U$
	\end{enumerate}
	then f has a unique fixed point $x^* \in U$, i.e $x^* = f(x^*)$. Moreover, the sequence $x_n = f(x_{n-1})$, with $ x_0 \in U$arbitrary, converges to $x^*$. 
\end{mythm}

\subsection{Inner product}

\begin{mydef}
	Let V be a vector space. $\langle\cdot,\cdot\rangle: V \times V \rightarrow \mathbb{R}(\mathbb{C})$ is called an inner product on V over $\mathbb{R}(\mathbb{C})$ if:
	\begin{enumerate}[label=(\roman*)]
		\item $\langle x, x \rangle \geq 0 $\:$\forall x \in \mathbb{R}(\mathbb{C})$
		\item $\langle x, x \rangle = 0 \Rightarrow x=0$
		\item $\langle \alpha x + \beta y, z \rangle = \alpha \langle x, z \rangle + \beta \langle y,z \rangle$ \: $\forall \alpha \beta \in \mathbb{R}(\mathbb{C})$, \: $\forall x,y,z \in V $
		\item $\langle x,y\rangle = \overline{\langle y,x \rangle}$ \:$\forall x,y \in V$
	\end{enumerate}
\end{mydef}

\begin{myex}
	\begin{enumerate}
		\item $\mathbb{R}$: \: $\langle x,y\rangle = \sum_{i=1}^{n}x_iy_i$
		\item $\mathbb{C}$: \: $\langle x,y\rangle = \sum_{i=1}^{n}x_i\overline{y_i}$
		\item $\langle f,g\rangle = \int_{a}^{b}f(x)g(x)dx$. Inner product of functions
	\end{enumerate}
\end{myex}

\subsection{Properties of a matrix}

\subsubsection{Positive definite}
A matrix is positive definite if $\langle Ax,x\rangle \geq 0$ ($\langle Ax,x\rangle = 0 \Rightarrow x=0$).\\
A matrix that is symmetric is positive definite if all its eigenvalues are positive.\\

To check if a symmetric (Hermitian in the complex case) matrix is positive definite, we can do two things:

\begin{enumerate}
	\item Check if all eigenvalues are greater than 0
	\item Check if all main minors are greater than 0 (i.e. det($A_{ii}$) $>$ 0)
\end{enumerate}

\subsubsection{Diagonally dominant}
I A is diagonally dominant, then A is regular, i.e. A is non-singular, i.e. det(A) $\neq$ 0.

\section{Condition of a problem}
\subsection{Condition analysis by solving linear systems}

We have the system $A\overrightarrow{x} = \overrightarrow{b}$ where $A \in \mathbb{R^{n,n}}, b \in \mathbb{R^n}$ and det(A) $\neq$ 0

\subsubsection{Condition number of a matrix A}
\begin{enumerate}
	\item $K_{abs} = \norm{A^{-1}}$
	\item$ K_{rel} = \norm{A}\norm{A^{-1}}$
\end{enumerate}

\subsection{Preconditioning}
Having a system $A\overrightarrow{x} = \overrightarrow{b}$, preconditioning means to multiply our system with a matrix $B \in \mathbb{R^{n,n}}$ such that K(BA)$<$K(A). We are solving $BA\overrightarrow{x} = B\overrightarrow{b}$, det(B) $\neq$ 0.\newline

The easiest preconditioning is to take B = diag($d_i$)










\end{document}
